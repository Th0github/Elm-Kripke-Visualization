\documentclass[12pt,a4paper]{article}
\usepackage{etex,datetime,setspace,latexsym,amssymb,amsmath,amsthm}
\usepackage{fancybox,dialogue,float,wrapfig,enumerate,microtype}
\usepackage{verbatim,xcolor,multicol,titlesec,tabularx,mdframed}

\usepackage[utf8]{inputenc}
\usepackage[pdftex]{hyperref}
\usepackage[margin=2cm,bottom=3cm,footskip=15mm]{geometry}
\parindent0cm
\parskip1em

\usepackage{tikz}
\usetikzlibrary{arrows,trees,positioning,shapes,patterns}
\usetikzlibrary{intersections,calc,fpu,decorations.pathreplacing}

\usepackage[T1]{fontenc} % better fonts

% Haskell code listings in our own style
\usepackage{listings,color}
\definecolor{lightgrey}{gray}{0.35}
\definecolor{darkgrey}{gray}{0.20}
\definecolor{lightestyellow}{rgb}{1,1,0.92}
\definecolor{dkgreen}{rgb}{0,.2,0}
\definecolor{dkblue}{rgb}{0,0,.2}
\definecolor{dkyellow}{cmyk}{0,0,.7,.5}
\definecolor{lightgrey}{gray}{0.4}
\definecolor{gray}{gray}{0.50}
\lstset{
  language        = Haskell,
  basicstyle      = \scriptsize\ttfamily,
  keywordstyle    = \color{dkblue},     stringstyle     = \color{red},
  identifierstyle = \color{dkgreen},    commentstyle    = \color{gray},
  showspaces      = false,              showstringspaces= false,
  rulecolor       = \color{gray},       showtabs        = false,
  tabsize         = 8,                  breaklines      = true,
  xleftmargin     = 8pt,                xrightmargin    = 8pt,
  frame           = single,             stepnumber      = 1,
  aboveskip       = 2pt plus 1pt,
  belowskip       = 8pt plus 3pt
}
\lstnewenvironment{code}[0]{}{}
\lstnewenvironment{showCode}[0]{\lstset{numbers=none}}{} % only shown, not compiled

% will the real phi please stand up
\renewcommand{\phi}{\varphi}

% load hyperref as late as possible for compatibility
\usepackage[pdftex]{hyperref}
\hypersetup{
  pdfborder = {0 0 0},
  breaklinks = true,
  linktoc = all,
}
\pdfinfoomitdate=1
\pdftrailerid{}
\pdfsuppressptexinfo15


\title{My Report}
\author{Me}
\date{\today}
\hypersetup{pdfauthor={Me}, pdftitle={My Report}}

\begin{document}

\maketitle

\begin{abstract}
We give a toy example of a report in \emph{literate programming} style.
The main advantage of this is that source code and documentation can
be written and presented next to each other.
We use the listings package to typeset Haskell source code nicely.
\end{abstract}

\vfill

\tableofcontents

\clearpage

% We include one file for each section. The ones containing code should
% be called something.lhs and also mentioned in the .cabal file.


\section{How to use this?}

To generate the PDF, open \texttt{report.tex} in your favorite \LaTeX editor and compile.
Alternatively, you can manually do
\texttt{pdflatex report; bibtex report; pdflatex report; pdflatex report} in a terminal.

You should have stack installed (see \url{https://haskellstack.org/}) and
open a terminal in the same folder.

\begin{itemize}
  \item To compile everything: \verb|stack build|.
  \item To open ghci and play with your code: \verb|stack ghci|
  \item To run the executable from Section \ref{sec:Main}: \verb|stack build && stack exec myprogram|
  \item To run the tests from Section \ref{sec:simpletests}: \verb|stack clean && stack test --coverage|
\end{itemize}


\section{Project}

The goals is to develop a web-based application that visualizes Kripke models. The application will leverage GraphViz library for graphical representations and Elm for frontend development. The backend will be a Haskell REST API. The backend will validate the Kripke models. Additionally, the project will have a help page to share how we have developed that application. 


\section{The MoSCoW table}
\begin{table}[!htb]
    \centering
    \begin{tabular}{p{0.9\textwidth}}
         \textbf{Must} \\
         \hline
         \begin{itemize}
             \item Model verification in Haskell Backend
             \item Visualize Kripke Models through GraphViz in Elm  
             \item Public announce in Haskell backend for build Kripke 
         \end{itemize}
        
         \\
         \textbf{Should} \\
         \hline
         \begin{itemize}
             \item Download the Kripke Models from Elm page in a svg file
             \item Data model the Kripke Model so it can be saved as json (text field)
             \item Form on frontend where user inputs agents,worlds and relations, which is parsed to a json
             \item Saving the kripke models in the database through the backend (mongoDB/firebase)
         \end{itemize}
        
         
         \\
         \textbf{Could} \\
         \hline
          \begin{itemize}
        \item Hosting the website on a platform (github or other platform)
         \item UI design
         \item Be able view all old uploaded models on platform
         \item Dockerize the application
         \end{itemize}
         \\
         \textbf{Won't} \\
         \hline
         \\
    \end{tabular}
    \caption{MoSCoW table}
    \label{tab:my_label}
\end{table}
% Haskell backend: Coen, Maurits 
% Elm front end: Halli, Thomas


\section{Objectives}
\begin{itemize}
\item Kripke Models in Elm: Develop tool to interpret and visualize Kripke models using GraphViz, allowing users to see a graphical representation of worlds and relations.
\item Interactive UI: Build the application in Elm to ensure a functional approach with strong emphasis on usability. 
\item Read me: Toggle a help page that will serve as the resource for the application was develop and how it can be used.
\end{itemize}


\section{Progress}
We have for the Beta phase created a way for a user to input a Kripke model, created a JSON representation of the Kripke model. Elm is able to make request to he fetch markdown documents to serve as. We have communicated the backend with the frontend such that the Kripke models. We can now save unvalidated Kripke models

Our current objective after the Beta phase is to represent the Kripke models as graph types. The backend will also be developed to validate the Kripke models. We would saving to saving the models to SQLite database. Dockerized the application and writing out help page will be done towards the end of the project.

%\section{Technologies Used}
%\begin{itemize}
%    \item Frontend: Elm is used  
%    \item Backend: Haskell - Scotty
%    \item Visualization: https://package.elm-lang.org/packages/elm-community/graph/4.1.0/Graph-GraphViz
%    \item Version control: https://github.com/Th0github/Elm-Kripke-Visualization.git
%\end{itemize}


\input{lib/Basics.lhs}

\input{exec/Main.lhs}

\input{test/simpletests.lhs}


\section{Conclusion}\label{sec:Conclusion}

Finally, we can see that \cite{liuWang2013:agentTypesHLPE} is a nice paper.


\addcontentsline{toc}{section}{Bibliography}
\bibliographystyle{alpha}
\bibliography{references.bib}

\end{document}
